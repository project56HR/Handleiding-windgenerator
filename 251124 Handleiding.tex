\documentclass[12pt, letterpaper]{report}
\usepackage{graphicx}
\usepackage{titlesec}
\usepackage{listings}
\usepackage{xcolor}
\usepackage{siunitx}
\usepackage[a4paper,margin=2cm]{geometry}
\usepackage[most]{tcolorbox}
\usepackage{fontawesome5}
\usepackage{hyperref}
\usepackage{colortbl}

\newtcolorbox{warningbox}{
	colback=red!10,
	colframe=red!70!black,
	coltitle=white!80!black,
	fonttitle=\bfseries,
	left=5pt,
	right=5pt,
	top=5pt,
	bottom=5pt,
	sharp corners,
	enhanced,
	title=\faExclamationTriangle\  Let op!,
}

\newtcolorbox{informationbox}{
	colback=blue!10,
	colframe=blue!70!black,
	coltitle=white!80!black,
	fonttitle=\bfseries,
	left=5pt,
	right=5pt,
	top=5pt,
	bottom=5pt,
	sharp corners,
	enhanced,
	title=\faQuestion\  Informatie,
}

\lstdefinestyle{mystyle}{
	basicstyle=\ttfamily\small,   % monospaced font
	keywordstyle=\color{blue},
	commentstyle=\color{green!50!black},
	stringstyle=\color{red},
	numbers=left,
	numberstyle=\tiny,
	stepnumber=1,
	numbersep=5pt,
	breaklines=true,
	breakatwhitespace=true,
	showstringspaces=false,
	frame=single,                 % adds a frame around code
	tabsize=4
}

\lstset{style=mystyle}


\titleformat{\chapter}[display]
{\normalfont\huge\bfseries}{}{0pt}{}


\begin{document}
	\title{Handleiding Mobiele Windgenerator\\
    \small Project 5-6 Hogeschool Rotterdam Technische Informatica\\
	\small Gemaakt in opdracht van MIND}
	\author{\small Gemaakt door Hannah Saunders (1093894), Berkin Demirel (1077286), Sem Hoogstad (1094004) en Mike Verkaik (1091369)}
	\date{19 November 2025}
	\maketitle
	
	\newpage
	\tableofcontents
	
	\newpage
	\chapter{Inleiding}
	Deze handleiding is gemaakt voor de Mobiele Windgenerator van MIND (Military Innovation by Doing) en bestaat uit drie delen. Het algemene deel, het technische deel (ICT) en de Installatiehandleiding. 
	\\
	\\
	Het algemene deel is bedoeld voor snelle oplossingen en algemene info, denk aan het aansluiten van de generator, eventuele kleine storingen/fouten oplossen, enz. 
	\\
	\\
	Het technische doel is bedoeld voor grootschalig onderhoud aan de apparatuur. Denk hierbij aan het updaten van software, aanpassen van verschillende instellingen enz. 
	\\
	\\
	De Installatiehandleiding is gemaakt voor wanneer de windgenerator vanaf nul wordt opgebouwd.
	
	\newpage
	\chapter{Algemene Handleiding}
	\section{Aansluiten en opstarten van de generator}
	De windgenerator wordt aangesloten aan de aansturingskast. Hier zitten vier verschillende poorten op. De stekker die van de windgenerator af komt past maar op een van deze poorten, namelijk de poort gelabeld \textbf{“GEN IN”}. 
	\\
	\\
	Schakel vervolgens het systeem in. Dit kan worden gedaan via de schakelaar \textbf{“SYSTEM”} worden gedaan. Zodra deze omgehaald is zal het scherm op de aansturingskast opstarten en zullen alle interne systemen opstarten. 
	\\
	\\
	Haal vervolgens de schakelaar \textbf{“GENERATOR”} op de aansturingskast om naar de RUN positie. De windgenerator is nu klaar om gebruikt te worden. Via de schuko aansluiting op de aansturingskast (gelabeld \textbf{“POWER OUT”}) kan 230v AC verkregen worden vanuit de batterijen. 
    \\
    
    % \includegraphics[width=1.0\textwidth]{}
    % \caption{afbeelding aansturingskast en benoemde poorten}
	
	\section{Opladen via Netspanning}
	De generator is ook op te laden via netspanning. Hiervoor zit de \textbf{“POWER IN”} poort op de aansturingskast. Hiermee kunnen door middel van een schuko naar IEC C13 snoer de interne batterijen worden opgeladen. 
	
	\newpage
	\section{Fouten, Foutcodes en andere problemen}
	\subsection{Geen 230v AC te verkrijgen}
	Dit betekent dat de batterijen niet voldoende opgeladen zijn en het systeem het ontladen van de batterijen uitschakelt. Laat de batterijen ophalen totdat er op het remote dashboard staat dat het systeem weer klaar is om energie te leveren. 
	
	\subsection{Verbinding verbroken tussen inverter/Batterij beheer }
	Zodra er een melding met de beschrijving \textbf{“Connection to BMS lost”} verschijnt op het scherm van de aansturingskast, betekent dit dat de verbinding tussen de inverter, de BMS en het beheersysteem verbroken is. Dit is op te lossen door te controleren of alle netwerkkabels nog verbonden zijn aan de correcte poorten. Zie ‘Zie figuur 2.1’ voor de handleiding/tekening hoe dit aan te sluiten is 
	
	\subsection{Batterijen laden niet op}
	Dit kan komen doordat een van de zekeringen is gesprongen in de aansturingskast. Deze zijn makkelijk te resetten. De zekeringen zijn te resetten door op de rode knop te drukken en vervolgens de hendel terug te duwen. Er zijn in de aansturingskast in totaal 4 zekeringen: 
	\begin{itemize}
		\item Een zekering tussen de generator en de aanvoer naar de charge controller 
		\item Een zekering tussen de charge controller en de batterijen 
		\item Een zekering tussen de Power In poort en de Inverter 
		\item Een zekering tussen de Power Out poort en de Inverter 
	\end{itemize}
	Zodra deze gereset zijn zal het weer mogelijk zijn om de batterijen weer op te laden met de windgenerator. 
	
	\begin{informationbox}
		Blijft de generator storingen melden? Neem dan contact op met MIND.
	\end{informationbox}
	
	\newpage
	\section{Schematische tekeningen systeem}
	\subsection{Elektrisch schema/algemeen schema}
	Hieronder een schematische tekening van de opbouw en aansluitingen van het interne systeem. 
	\begin{figure}[h]
		\centering
		\includegraphics[width=1.0\textwidth]{Elektrisch-schema.png}
		\caption{Elektrisch schema van de windgenerator}
		\label{fig:ElektrischSchema}
	\end{figure}
	
	\subsection{Aansluiting Netwerkkabels}
	Op de Victron VE.Bus BMS v2 zitten 2 netwerkpoorten. De \textbf{“Remote Panel”} poort en de \textbf{“Multiplus/Quatro”} poort. Verbindt de \textbf{“Remote Panel”} poort met een van de \textbf{ BUS”} poorten op het beeldscherm wat op de aansturingskast gemonteerd zit. Verbindt de \textbf{“Multiplus/Quatro”} poort aan de Multiplus Inverter. Hierop zit maar 1 netwerkpoort.
	
	\newpage
	
	\chapter{Technische Handleiding}
		\begin{warningbox}
		Dit deel van de handleiding is alleen voor medewerkers van MIND. Zorg daarom ook altijd dat er iemand van MIND aanwezig is als de onderstaande handleiding wordt uitgevoerd.
	\end{warningbox}
	\section{Raspberry PI}
	\subsection{Flashing PI-OS}
	\begin{itemize}
		\item Stap 1: Download de Raspberry PI Imager en installeer deze op een computer.\\
            Het programma voor het instaleren van de imager is te vinden op:\\ \url{https://www.raspberrypi.com/software/}\\
            Volg vervolgens de instructies van het programma.
		\item Stap 2: Stop een Micro-SD kaart van minimaal 32gb in de computer (Niet de PI) 
		\item Stap 3: Open Raspberry PI Imager. Selecteer hier de SD kaart en vervolgens het operating system wat je wilt installeren (Raspberri PI OS) 
		\item Stap 4: Klik op \textbf{“Next”} en vervolgens op \textbf{“Edit Settings”}
		\item Stap 5a: Onder \textbf{“General”} zet je de naam op \textbf{“raspberry.local”} (hoofdlettergevoelig). Stel zelf de gebruikersnaam en het wachtwoord in. Let op! Dit zijn de inloggegevens van de Raspberri PI zelf en ook van de SSH service.
		\item Stap 5b: Onder \textbf{“Services”} zet je de SSH Service aan. Zorg ook dat \textbf{“Use password as authentication”} checkbox aangevinkt staat. 
		\item Stap 6: Klik op \textbf{“Save”}, en vervolgens op \textbf{“Next”} om te starten met het flashen van Raspberry PI OS.  
		\item Stap 7: Zodra de Raspberry PI Imager klaar is met het flashen van Raspberry PI OS kan je de Micro-SD kaart veilig ejecten.
	\end{itemize}
	
	\newpage
	\section{Raspberry pi terminal}
    \subsection{Opening a terminal in the pi}
    \begin{itemize}
        \item Stap 1: verbindt een laptop via een netwerk kabel aan de Pi.
        \item Stap 2: Open de terminal op de laptop.
        \item Stap 3: type het volgende in de terminal en druk op enter.
         \begin{lstlisting}
			ssh project56@192.168.1.101
		\end{lstlisting}
        \item Stap 4: vervolgends zie je dit:
        \begin{lstlisting}
			project56@192.168.1.101 password:
		\end{lstlisting}
        Hier voer je in het wachtwoord: project56 (het wachtwoord dat je in typt wordt niet getoond in de terminal)
    \end{itemize}
    
	\section{Docker}
	\subsection{Installatie van de Docker}
	\begin{itemize}
        \item Stap 0: open de terminal aan de hand van de instructies in hoofdstuk 3.2.1.
		\item Stap 1:  Verifieer dat Docker is geïnstalleerd doormiddel van het uitvoeren van: \begin{lstlisting}[language=Bash, caption={Verifying that docker is installed}]
			docker -v
		\end{lstlisting} Dan moet je iets zien van Docker version 27.1.1, build 6312585. 
		\\
		Zie je dit niet? Ga dan door naar stap 2. Zie je dit wel? Ga dan door naar Stap 3.
		
		\item Stap 2: Installeer docker via het officiële installatie-script. Download het script doormiddel van het uitvoeren van:
		\begin{lstlisting}[language=Bash, caption={Installing docker with the terminal}]
			curl -fsSL https://get.docker.com -o get-docker.sh
		\end{lstlisting} 
		Docker Engine, Docker CLI en Containerd worden dan automatisch geīnstalleerd.
		
		\item Stap 3: Voeg je gebruiker toe aan de docker-groep zodat je geen $sudo$ meer hoeft te gebruiken. Doe dit doormiddel van:
		\begin{lstlisting}[language=Bash, caption={Adding user to the docker group}]
			sudo usermod -aG docker $USER
		\end{lstlisting} 
		Hierna moet je opnieuw inloggen of de Raspberry PI opnieuw opstarten (optie 2 wordt aangeraden)
		
		\item Stap 4: Test of Docker werkt. Dit kan worden gedaan doormiddel van het uitvoeren van:
		\begin{lstlisting}[language=Bash, caption={Testing if docker is working}]
			docker run hello-world
		\end{lstlisting} 
		Je moet een bericht krijgen waarin staat dat Docker correct werkt.
		
	\end{itemize}
	
	\subsection{Folder van USB-Stick halen en Docker build uitvoeren}
	\begin{itemize}
        \item Stap 0: Open de terminal aan de hand van de instructies in hoofdstuk 3.2.1.
		\item Stap 1: Steek de USB-stick in de Raspberry PI. Deze wordt vaak vanzelf gemount onder $/media/pi/NAAMUSB$
		\item Stap 2: Controleer of de USB-stick zichtbaar is. Dit kan worden gedaan door 
		\begin{lstlisting}[language=Bash, caption={Checking if USB stick is working}]
			ls /media/pi
		\end{lstlisting}
		 uit te voeren in de terminal. Je zou dan de naam van de USB-stick moeten zien.
		 
		 \item Stap 3: Navigeer naar de USB-map.
		 \begin{lstlisting}[language=Bash, caption={Navigating to folder on USG stick}]
		 	cd /media/pi/NAAMUSB
		 \end{lstlisting} 
		 Daar staat de folder met het Docker project.
		 \item Stap 4: Kopieer de projectmap naar de Rappberry PI.
		 \begin{lstlisting}[language=Bash, caption={Copying directory to the PI}]
		 	cp -r PROJECTMAP /home/pi/
		 \end{lstlisting} 
		 Het wordt afgeraden om het project vanaf de USB stick te bouwen ivm de snelheid van de stick en het gevaar dat deze tijdens de installatie kan stoppen met werken. Vandaar dat de projectmap naar de PI verplaatst moet worden.
		 
		 \item Stap 5: Ga naar de gekopieerde projectmap
		 \begin{lstlisting}[language=Bash, caption={Navigating to the copied project directory}]
		 	cd /home/pi/PROJECTMAP
		 \end{lstlisting} 
		 Hier staat de Dockerfile en alle andere projectbestanden
		 
		 \item Stap 6: Start de Docker build
		 \begin{lstlisting}[language=Bash, caption={Building the Docker container}]
		 	docker build -t MobileGenerator .
		 \end{lstlisting} 
		 
		 \item Stap 7: Controleer of de Docker build is gelukt. Dit kan worden gedaan met het uitvoeren van
		 \begin{lstlisting}[language=Bash, caption={Checking if the container has been built succesfully}]
		 	docker images
		 \end{lstlisting} 
		 Hier zou nu de naam van de container tussen moeten staan ($MobileGenerator$ in dit geval, tenzij een andere naam is ingevuld)
		 
		 \item Stap 8: Nu kan de container worden gestart. Dit kan worden gedaan door het volgende uit te voeren
		 \begin{lstlisting}[language=Bash, caption={Starting the Docker container}]
		 	docker run MobileGenerator -p'3001:3000/tcp' -p '32848:32848/tcp -v' /home/pi/PROJECTMAP ':'/workspace':'rw' -d
		 \end{lstlisting} 
		 \begin{informationbox}
		 	-d is niet verplicht. Dit is alleen nodig als de container op de achtergrond moet draaien (wat in de meeste gevallen zo is)
		 \end{informationbox}
		 	
		 	\item Stap 9: USB-Stick veilig verwijderen (belangrijk!)
		 	\\
		 	Zodra alles gekopieerd en opgebouwd is moet de USB-stick veilig verwijderd worden. Dit kan door het uitvoeren van
		 	\begin{lstlisting}[language=Bash, caption={Safely removing the USB-stick}]
		 		sudo unmount /media/pi/NAAMUSB
		 	\end{lstlisting} 
		 	Hierna kan de USB-stick uit de PI verwijderd worden
	\end{itemize}
	
	
	\newpage
	
	\chapter{Installatiehandleiding}
	\begin{warningbox}
		Dit deel van de handleiding is alleen voor medewerkers van MIND. Zorg daarom ook altijd dat er iemand van MIND aanwezig is als de generator opnieuw wordt opgebouwd
	\end{warningbox}

	
	\section{Onderdelenlijst}
	\begin{itemize}
		\item 1x Superwind SW353
		\item 1x Superwind SW353 Run/Stop schakelaar
		\item 1x SCR Marine 12 Charge Controller
		\item 1x 0,35 Ohm Load Resistor
		\item 1x Victron Energy Multiplus 12 / 800 / 35
		\item 2x Victron Energy Lithium 12.8V-50Ah Smart LiFePO4
		\item 1x Victron Energy Ekrano GX
		\item 1x Victron Energy VE.Bus BMS V2
		\item 2x 30A zekering
		\item Minstens 15m \SI{6}{\square\milli\meter} draad (15m per kleur)
		\item 30cm \SI{2.5}{\square\milli\meter} draad (Rood)
		\item 3x Cat 5e (of hoger) UTP kabels
		\item 1x Stijgerbuis (60.3mm Diameter, minimaal 2m hoog)
		\item 1x Stijgerbuis Voetstuk (60.3mm Diameter)
		\item 1x Verzwaarde Voet
		\item 1x Raspberry PI 5
		\item 1x Raspberry PI 5 4g Hat
		\item 1x \textit{\textbf{Unmanaged}} Netwerk Switch (min. 5 port)
	\end{itemize}
	
	\newpage
	\section{Bekabeling systeem}
	\subsection{Elektra}
	Hieronder twee figuren van hoe de elektra moet worden aangesloten. Het is belangrijk dat alles met \SI{6}{\square\milli\meter} draad wordt aangesloten (tenzij anders aangegeven).
	\\
	\\
	Tussen de output van de Victron Energy Multiplus 12 / 800 / 35 zit een EATON PLN6-B16/1N automaat. Deze is bevestigd aan een DIN rail, samen met de andere 2 zekeringen.
	\begin{figure}[h]
		\centering
		\includegraphics[width=1.0\textwidth]{ElektraBatterij.png}
		\caption{Elektrisch Schema van de Batterijen}
		\label{fig:ElektraBatterij}
	\end{figure}
	\begin{figure}[h]
		\centering
		\includegraphics[width=1.0\textwidth]{ElektraGenerator.png}
		\caption{Elektrisch Schema van de Generator}
		\label{fig:ElektraGenerator}
	\end{figure}
	\begin{warningbox}
		Raadpleeg voor het installeren en aansluiten van de windgenerator de handleiding
		voor de Superwind SW353. Deze is te vinden op de website van SuperWind: \url{https://www.superwind.com/fileadmin/user_upload/Downloads/350/SW350-II_Operation_Manual.pdf} en zal ook	worden meegeleverd met het product zelf.
	\end{warningbox}
	\newpage
	\subsection{Netwerk en Data}
		\begin{figure}[h]
		\centering
		\includegraphics[width=1.0\textwidth]{DataEnNetwerk.png}
		\caption{Schema van Netwerk en Data}
		\label{fig:DataEnNetwerk}
	\end{figure}
	\begin{informationbox}
		De datakabels van de 2 batterijen zitten standaard aan de batterij vast. De batterijen moeten aan elkaar doorgelust worden met een male en female stekker. De andere twee overblijvende stekkers moeten aan de VE.Bus BMS V2 worden verbonden.
	\end{informationbox}
	
	\section{Hardware en monteren steigerbuis}
	Monteer de steigerbuis voet aan een stevige en verzwaarde basis. Vervolgens kan de steigerbuis in de voet worden geschoven en worden vastgeschroeft met een inbussleutel (type 8).
	\\
	De kop van de windgenerator wordt op 4 punten aan de steigerbuis vastgeschroeft. Hier moeten 4 $... mm$ gaten voor worden geboord, die elk op 90 graden van elkaar in de buis moeten worden geboord.
    \\
    
    \newpage 
    \chapter{Changelog}
    \begin{table}[h]
    \arrayrulecolor{black}
        \begin{tabular}{|c|c|c|}
        \hline
            Date & version & description\\
            \hline
            25/11/2015 & 1.0 & creatie document\\
            \hline
            01/12/2025 & 2.0 & update manual\\
            \hline
            02/12/2025 & 3.0 & update\\
            \hline
            26/01/2026 & 3.1 & updates uitgevoerd aan de hand van feedback\\
            \hline
        \end{tabular}
    \end{table}
    
	
	
\end{document}
